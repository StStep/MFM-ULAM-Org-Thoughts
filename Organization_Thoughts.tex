\documentclass[article,12pt,oneside]{memoir}

\usepackage[hidelinks]{hyperref}

\tightlists

\usepackage{enumitem}
\setlist{before=\vspace{-0.5\baselineskip}, after=\vspace{-0.5\baselineskip}}

\setulmarginsandblock{1in}{1in}{*}
\setlrmarginsandblock{1in}{1in}{*}
\checkandfixthelayout

\setlength{\parindent}{0em}
\setlength{\parskip}{1.5ex}

\title{ULAM and MFM Organizational Thoughts}
\date{August 2, 2017}
\author{R. Stephens Summelrin}

\begin{document}

\maketitle
\tableofcontents

\pagebreak
\chapter{Overview}

This document is a semi-formal collection of my notes and thoughts concerning organizing calculations and data for systems like The Movable Feast Machine (MFM).
I've mostly been focusing on allocation, organization and data routing; treating them as strongly linked topics.

My research went down two major paths, which are described below.
I haven not implemented either of these systems, so what follows is almost entirely theory.
I do have an bubble allocation demo, which is exciting but not super relevant.
It's here: \url{https://www.youtube.com/watch?v=diiUx5CkXa4}.
Note that later versions have area balancing function, which isn't present in the video.


\section{Cell-Organism Hierarchy}

This organization is hierarchical and is on a large scale, where elements make up the matter that then becomes organizational units.
This idea is driven by the 'Law of physics' being interpreted as basic parts that are used to build up to organizational units that have insides outsides and homeostasis.
This leads to elements forming into parts of organisms, which are the lowest organization level that have homeostasis.
Continuing with biological inspired terminology, organisms would then interact together forming an ecosystem.
Ecosystems would have multiple copies of the same organisms, either competing or supporting one another.

This area is the one I spent most of my time in and have the most material for.
Ultimately, I determined that this system is unnecessary complicated.
While it's fun to come up with ideas and organizations, a lot of this complexity doesn't seem to lead to worth the sometimes vague benefits.


\section{Agent}

This organization handles the system as a collection of agents.
Elements are generally individual agents, and there will be a lot of negative space.
It will have little to no hierarchy.
An element represents a general agent.
This organization is inspired by city management games.
I imagine centers of activity with data processing setup along data routes through the system.
Akin to cities importing and exporting goods along roads of varying size.

This idea is both simpler and less developed.
It feels much more manageable so far, but again note that I have not really implemented either of these systems, so it's just manageable in theory.


\pagebreak
\chapter{Cell-Organism Hierarchy Organization}

This system is heavily based around allocation, especially of area.
Routing revolves around navigating hierarchy boundaries, while moving up to higher hierarchies for traveling longer distances.
There would be specialized structures for moving data on the top, ecosystem, level.

I refer to this idea sometimes as the 'River and Trees'.
The river being the primary routing system for the ecosystem, with the trees being the organisms on the banks that handle and produce data structures.
The river would both loop around and flow into and out of the the application area using some kind of I/O organisms; capping the boundaries.
There would other entities to handle life-cycle, such as breaking down or removing detritus elements.
The Trees themselves would be grown from stem-cell like collections of elements, or seeds.
These groups would circle throughout the river, maintaining a certain distribution, and take root along the banks under certain conditions.
This would lead to a dynamic distribution of organisms, that can die and be replaced according the distribution of data structures or signal elements.


\section{Allocation}

Space is a major resource in this organization and the management of this space is generally referred to as allocation.
This was my initial area of study into this system, and forms the base and direction of a lot of my further research.
This is broken into two areas.

The unified area is a dynamic allocation that takes as much space as it can, up to a size.
It would theoretically be the most efficient use of space, but it is difficult to manage.
It works well with simple problems, but doesn't scale well, due to a large percentage of area being dedicated to allocation.

The Segmented Allocation is an allocation of a number of fixed segments.
It is simpler to manage, and a smaller percentage of total area is for allocation logic.
This allocation type seems more reasonable, and I turned to this after having issues scaling the unified allocation method.


\subsection{Unified}

Unified allocation is allocation done an a site by site basis, and I broke it into two types.


\subsubsection{Bubble Allocation}

Bubble Allocation creates a circle of a max radius if given infinite room.
It allocates using a max radius from a center point, locating a centroid of a space if constrained.
This can be done by having edge elements of the allocation maintain distance from the center though some life-cycle.
I do this by having the edges dissolve if they are surrounded by inside, die if they are alone, and stay strong if they are on the edge or surrounded by foreign elements.
A demo can be found here: \url{https://www.youtube.com/watch?v=diiUx5CkXa4}.


\subsubsection{Path Allocation}

I never implemented path allocation as opposed to bubble allocation.
It would creates a path of a given max width and length if given infinite room, with the goal being the allocation of a path of a minimum width.
It would allocations from an edge outwards using a max width and max length, while ideally bending to use up available space.

Some implementation ideas follow:

\begin{description}
	\item[Edge Idea] Build layer by layer by active edge
	\item[Blow Bubbles Idea] Use of multiple bubble allocations along path with spacing logic
	\item[Grow Noodle From Base Idea] Inflate from edge
	\item[Diffusing String Idea] Reel out a string of diffusing elements that thicken
	\item[Bristle Chain Idea] A chain of elements with sensory parts, bristles, detecting available space; tries to grow to maximize space.
\end{description}


\subsection{Segmented}

World would be made up of allocation grid, made of mostly equal size segments.
A segment would be only allocated for a single purpose, with some being dedicated t routing.
Routing could be from organism to organism, or be a general purpose route: the River and Trees idea.

Allocation segments would have a control portion and a useful portion.
The control portion would track what the segment is allocated for, and handle segment initialization, clearing, and re-allocation.


\section{Codons}\label{codons}

Codons are a data storage method that takes space, and has redundancy checks for a robust way of maintain data across long distances.
This storage method is largely influenced by DNA storage, hence the name codon.
A Codons consist of three Pairs of elements, with 64 total combinations.
An individual pair simply needs to be an identifiable combination of two elements.
A codon translate to an A, C, G or T if at least two of the three pairs are A, C, G or T respectively.
This system allows for operation to be perform on all three pairs independently, where a translation can happening afterwards to remove unlikely resolutions.
For error handling, invalid pairs or codons could be dropped.

Codons would be strung together to form values. With special codons designated as Stop and Start codons.
It would essentially form a base 4 storage system.

The combination breakdown is as follows:
\begin{itemize}
	\item A - two of three are A (10)
	\item C - two of three are C (10)
	\item G - two of three are G (10)
	\item T - two of three are T (10)
	\item Start - two of three are A and C (12)
	\item Stop - two of three are G and T (12)
\end{itemize}


\section{Organism DNA}

This sections describes what I called Organism DNA, which would essentially be organism blueprints.
It would store information like:

\begin{itemize}
	\item Size
	\item Reproduction Rate
	\item Lifespan
	\item Reproduction Requirements
	\item Growth Requirements
	\item Life Requirements
\end{itemize}

From DNA you should be able to grow another of that organism, and it would make up most of what I refer to as seeds or stem-cells.
The DNA itself would use something like Codons described in Section \ref{codons}.
The container could act as a reader, to save space, and would likely need elements for reading, checking for mutations, repairing mutations, and knowing when to grow or reproduce or die.

\section{Calculation}

Calculation isn't something I spent a lot of time on.
I assume parts of an organ would handle calculations.
One idea was to have an organism have a loop where input goes through gates where operations would take place, and outputs go back into the environment.
This works well with the Codon data method described in Section \ref{codons}.



\pagebreak
\chapter{Agent Organization}

Idea
Perhaps thing more cooperative citizens rather than cell organization?
Like ant swarms, agents build structures to perform calculations
Demo that does a TCP router using a colony of agents
Little to no barriers
Layering Through Site Alternation
Have a distance measureing layer by altering every-other site with a communication element
Allows for simultaneous direction finding elements and operational elements
Dynamic Sizes vs Static Sizes
If factories are of a static size then less runtime calculations need to occur
Distance vs Direction
Direction to a thing can be found but if it is, than it cant show distance while also being indefintily scalable
Question of wether distance or direction is stored using a communication layer
Distance can be used by resetting to MAX VAL when 0 is reached, leading to increasing values being direction for a thing
Additionaly, direction requires fewer bits for fidelity, possible with as little as 2 bits
Subtractive growth
Grow by digging into dirt
Dirt collapse acts as a way to cleanup old stuff
Agents
Build structures
Such as communication towers to commicate over long distances
Or roads?
Not restricted in movement
Structures
Summary
Structures have modular sections
Factories
Perform operations
Customize
Factory Plans
Growth
use dummy data to find good positioning?
Lead to growth of factories that produce desired input
Farms
Generate data
Uses
RNG
Energy source that can be used to limit growth
Ports
Imports/Exports data
Customize 
Port List
Factory Plans
Growth
Grow without prompt on I/O
Lead to growth of factories that produce desired input
Roads/Towers
Move data
Important structures need attachment point
Growth
Built to meet each other
Unused paths die off
Expect primary path with connected ancillary paths



\pagebreak
\chapter{Miscellaneous}

\section{Saving Space}


Flicker Element

Flicker between elements conditionally if there is a code size limit




\end{document}