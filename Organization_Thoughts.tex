\documentclass[article]{memoir}

\title{ULAM and MFM Organizational Thoughts}
\date{August 2, 2017}
\author{R. Stephens Summelrin}

\begin{document}

\maketitle
\tableofcontents


\chapter{Overview}

I've mostly been focusing on allocation, organization and data routing. 
Treating them as all highly linked.

Spent time going down different paths:



\section{Cell-Organism Hierarchy}

Elements are stuff
Heavily allocation based
Need boundaries
Hierarchical, River and Trees
Macro and micro design levels


Pretty much most of my prior work
Law of physics are basic parts that are used to build up to organizational units that have insides outsides and homeostatis
Well defined boundaries
Heirachical organization
Focus on moving data to destination cells



\section{Agent}

Elements are agents
Open floor plan, so to speak
Negative space okay
Single design level



\chapter{Cell-Organism Hierarchy Organization}


Specifications
General restrictions being followed.
Element Restrictions
Note: There are 71 usable bits for element instance members, and 32 bits for quarks.
Personal Restrictions
Currently, the event window look-ups are restricted to adjacent sites, sites 1 through 4 as a HW implementation feasibility measure.
Allocations Types
Summary
The desired requirements for allocation, size management, systems are the following:
Containing a space with a specified shape
Able to specify a max allocation
Separates between different allocation IDs and from unknown elements
Growth of allocated space tries to fill available space
Deal gracefully with inconsistencies in shape of available space
Kill allocation process if unfeasible
Demos
[ULAM - Bubble Allocation Demo]([https://www.youtube.com/watch?v=diiUx5CkXa4](https://www.youtube.com/watch?v=diiUx5CkXa4))
Bubble Allocation
Summary
Creates a circle of a max radius if given infinite room
Allocates using a max radius from a center point
Locates the centroid of a space if constrained
Standard allocation on display in github repo.
Development Process
The chosen bubble allocation process went through multiple development stages. The selection process involved multiple development groups, with Group 4 being chosen. More details in past commits.
Group 1 CANCELED 
Crystalline growth, limited by inconsistencies in available area
Group 2 CANCELED 
Initial amorphous growth attempt, unstable
Group 3 CANCELED 
Pulsing growth attempt, ran into growth instability, fuse problem
Group 4
Based on group 2, with a logic pass-over to improve stability
Even Distribution of Nuclei of same ID
Use barriers between nuclei of same ID
every nucleus/stem have a unique ID separate from type
transparent barriers form between nuclei of the same type
allocations will join up together upon death/finalization
if two of same ID meet, they should merge
emitters transmit through transparent barriers?
Future Work
Currently, Cytoplasm based distance measuring means internal non-Cytoplasm elements will alter the overall shape of the allocated space
Nucleus auto-centering works with simple organizations, current issues:
Edges of world are preferable to auto-center, going to expect edges to exist for now
Elements
Nucleus
This element initiates site allocation, by creating Cytoplasm elements, and setting their dist data member.
Has an ID, that is bestowed on Cytoplasm children.
The ID is used to separate allocated sites by purpose.
Two nuclei of differing ID will not merge, and will stay distinct.
Two nuclei of the same ID will share allocated sites.
A nucleus will die if it is surrounded by Membrane elements.
Nuclei currently attempt to auto-center themselves within spaces, see Future Work for progress concerning this.
Parameters:
maxDist - This parameter determines how much space is allocated.
Cytoplasm
This element serves as the primary mechanism for size management.
Keeps track of a distance value, that decrements going outwards from the nucleus, that is used to determine distance.
The life-cycle of Cytoplasm is strongly connected to the Membrane life-cycle.
Membrane
This element represents the boundary of allocated space.
This element forms as Cytoplasm elements contact unknown elements, Cytoplasm of a differing ID, or the max allocation range.
Path Allocation
Summary
Creates straight path of a given max width and length if given infinite room
Goal is the allocation of a path of a minimum width
Allocations from an edge outwards using a max width and max length
Ideally, will bend to use up available space
Development Process
Below are proposed development directions that this allocation can be accomplished by.
Edge Idea
WAITING
Build layer by layer by active edge
The edge being built changes from cycle to cycle based on open sites
Blow Bubbles Idea
Using multiple bubble allocations
The centers of each bubble allocation form the path
Could have a distance restriction from previous bubble
While trying to separate self from other bubbles
Retracts if previous bubble moves too far away
Use transparent barriers and exciters to track nuclei separation
Variables
N - number of bubbles
W - Width of path
Path length ~ W(N/4 + 1/2)
Bubble radius ~ W/2
Bubble separation ~ W/4
Grow Noodle From Base Idea
WAITING
Have forward edge handle collision and bending?
Bristled Chain Idea
IN-PROGRESS

Summary
A chain element goes out with a priority directions.
There are ‘bristles’ on either side that detect the distance to an obstacle
And it turns based on the distances measured on either side
If the minimum distance is ever met, than the chain element will die
Will marking its direction as bad in its parent chain
If chain under min chain length, then it will retract and try again, making different direction choices
In this way, the chain will crawl backwards when a link dies
Hopefully finding a more optimal path
Could optimize by increasing amount of info kept about bad paths
Variables
Priority Direction
Min bristle distance
Max bristle distance
Max chain length
Min chain length
Bad Direction [4]
Logic
Pointing Priority
if d < Max
Turn Away onto not bad non-priority direction
elif d < Min
Die and mark as bad
else
Keep Straight
Else
if d < Max
Go current direction
elif d < Min
Die and mark as bad
else
Return to priority direction
Implementation Plan
Implement simple single width path
Only detects adjacent sites to chain, will detect non-adjacent chains and obstacles
Implement bristles, only detect on obstacles
Modify detection to be half of width, using new bristle elements Base on bubble centering logic
Implement prior path bristles detection by current bristles
Detect bristles from previous chains so the path will maximize turning radius and reduce bristle overlap
Add in optimization’s such as keeping more bad path info
Add in logic and more data to allow for path assumptions to reduce dead-end path attempts
Future Work
Not yet implemented
Elements
Path Stem
Chain
Bristle
Organization Levels
The following are the requirements for organization:
Relative addressing levels
Absolute addressing is not possible
Perform some kind of life-cycle management
Current Levels include:
Organ Layer
Lobe Layer
Leaf Layer
Hierarchical Loop Organization
This forms the general organization of an organ. A hierarchical loop, where a central loop has a controlled distribution of contents. This distribution is then propagated through adjoining loops, indirectly.
Benefits
Higher throughput than a melange, more directed. Encapsulation of functionality. Protect information better?
Organization Summary
Organ
Bubble allocation
Variables
Max Radius
Organ ID
determine leave/stem IDs
Lobes (generic)
Path allocation
Variables
base width
max length
Leaves
Path allocation
Variables
Base width
max length
Stem ID
Determines filters
Notes
Organ
Hierarchical loop top-level
Primary address level
Has an allocation bubble
Basic I/O ports
Central kernel/node
Connects to lobes and I/O
Lobes have width, determines leaf height
Coil out into available space
Fit as many as possible with given width
Leaves
Grow within lobe wall
Circulated through system and embed where there is open space
simple loop
has base width
stem forms center
Filters grow from stars
perform data processing
Growth and Life-cycle
Leaf and Lobe Growth
Could be similar
Has base width
Pushes into growth medium
Stem path is important part of lobe growth
leaves can cross lobe boundaries
both could embed and grow?
Organ Life-cycle
1. Organ stem exists
2. Performs bubble allocation
3. stem -> kernel
4. Grows I/O lines
5. Kernel inserts lobe stems
6. Kernel inserts leave stems
7. Open for business
8. Homeostasis
Lobe Life-cycle
1. Lobe stem exists
2. embed self in kernel wall
3. perform noodle allocation
4. Homeostasis
Leaf Life-cycle
1. Leaf stem exists
2. Embeds self in lobe wall
3. Performs noodle allocation
4. Grows filters
5. Homeostasis
Homeostasis
Allocation
Die if allocation issues found
Stem cells in permanent allocation, constant circulation
Promotes regrowth of failed leaves and lobes
If leave type missing or in short supply, cull population and replace
Determine percentage of each
Dynamically handle % based on input
starving leaves w/constant stem circulation?
Dynamic regulation of stems in kernel
Circulation
Kernel has a circular flow
I/O have straight flow
Gating
Gates change addressing levels? Could leaf data only be valid within organs? Needing a special conversion to be viable on the organ addressing level?
Organ ID based
Organ Gate
White-list data
Kernel Gate
White list data
Leaf ID Based
Leaf Gate
White-list data
Dynamic
Lobe Gate
Dynamically white-list based on contained leaf IDs
IDs
Organ IDs
Input data
Output data
Leaf ID distribution
Allocation size
Lobe width and length
Leaf ID
Input data
Filters -> Output Data
Width and Length
Development
Block out hierarchy
Implement Path Allocation
Melange Organ
Organs contain a mixture of operator elements. Much simpler structure than Hierarchical loop organization. Operator elements organize themselves into stratification, move around depending upon if they have available input elements.
Routing would be to an organ ID first than an operator.
Benefits
Simple
Inner Organization
Should only move when seeing data? Try to organize based on certain ideas?
????



\section{DNA}
Stores
Size
Reproduction Rate
Lifespan
Reproduction Requirements
Growth Requirements
Life Requirements
Format
Container acts as reader
Box container
3x3 box contains matching codons and center read row
The program for an organism
Needs to be easily duplicated
Omnidirectional
Organelles
Combinations of actions and values
Types
Membrane Maintain
Unknown​/Error Destroy
DNA Repair
Reproduction Monitor
Lifespan Monitor
Growth Monitor


\section{Allocation}

Tiles are resource
Segmented vs Unified

\subsection{Segmented}

allocation a segment at a time, static borders

+ World is made up of allocation grid, made of mostly equal size segments
+ Segments allocate in entirety 
+ Segment could be routing, or processing, etc


+ Dedicated Operation Routes
- perform allocation based on desired operations
- this would make it very regimented
- would have dedicated paths from one operator to another
+ General Purpose Route.
- would have multiple things in each route
- would need some means of pulling things from and adding to
- might want to use a heirarchical organizational system?


Segmented Allocation
Routing
Dedicated
Growth
Operator tracks input signal
Growth Mode
Stops operation
Allows for confirmation of calculation chain
Directly based on equation
Linear
Interstitial space where signals can traverse
Multi-purpose
Growth
Common Route grows first
Common route is seeded
Connect inputs and outputs
Seeds embed into walls
Common route
Need another org level
Nonlinear
Operator pulls from common route
Could have linear routes
Train style?
Data
Pattern
Based off DNA
Pairs
Smallest unit
Error resistance irt single unit
Partners
A and T
G and C
Codons
Three pairs
Interpreted by operators
Forms were resistance irt sequential pairs
Equate To
Start
Stop
A
T
G
C
Allocation
Route
Operators
Static size
Operator has specific size it wants
Useful if arb num operators deployed
Percentage
Allocate in halves quarters?
Useful if specific number of operators deployed
Determination
How are operators chosen
Input data directly leads to operators
Special Input signal defines relationship​
Determined by Organ Equation
Operators
Allocate multiple contiguous segments
Input and output segment walls
Organization Levels
Laws of Physics
Singular, used by all organisms
Ecosystem
Application level
Determines
overall structure
What inputs go to organs
What leads to outputs
Definition
Defined by homeostasis organisms
Homeostasis
Manages organisms distribution
Organisms are replaceable
Flexible organization required
Environment
Organisms + Environment = Ecosystem
Organism
Level of primary replacement
Rigid organization allowed
Organ
Generic, in multiple organisms
Takes input, performs operation
Organ Addressing I/O
Building blocks of organisms
Is not replaced


\subsection{Unified}

Circle Allocation

Path Allocation

blowing bubbles

+Stem blows bubble with ID N
+Bubble n blows bubble with ID n-1 if no n-1 doesn't exist
+Bubble n dies if n+1 ID can't be found 

diffusing strings

Path Allocation Through Chipping

+ Each circle allocation takes out a chunk
+ Surface that can be eaten is marked
+ 'Mark' is a line that self adjusts
+ Maintain​s a min and and max length
+ dissolves like acid
+ 3 stage swap


\section{Calculation}


\subsection{Melange}

+ Whenever operand found, the operator moves left if not moves right
+ Should lead to layering of operators in reasonable manner?

Melange of Operators Use data as notepad
Can keep track of past calculations to have a running average on data and calculations

\section{Codons}

Codons consist of three Pairs, 64 combinations
The sense is used for encoding
- Math occurs independently for pairs within codon
- translation happens before math -> error refresh

+ A - two of three are A (10)
+ C - two of three are C (10)
+ G - two of three are G (10)
+ T - two of three are T (10)
+ Start - two of three are A and C (12)
+ Stop - two of three are G and T (12)



\chapter{Agent Organization}

Idea
Perhaps thing more cooperative citizens rather than cell organization?
Like ant swarms, agents build structures to perform calculations
Demo that does a TCP router using a colony of agents
Little to no barriers
Layering Through Site Alternation
Have a distance measureing layer by altering every-other site with a communication element
Allows for simultaneous direction finding elements and operational elements
Dynamic Sizes vs Static Sizes
If factories are of a static size then less runtime calculations need to occur
Distance vs Direction
Direction to a thing can be found but if it is, than it cant show distance while also being indefintily scalable
Question of wether distance or direction is stored using a communication layer
Distance can be used by resetting to MAX VAL when 0 is reached, leading to increasing values being direction for a thing
Additionaly, direction requires fewer bits for fidelity, possible with as little as 2 bits
Subtractive growth
Grow by digging into dirt
Dirt collapse acts as a way to cleanup old stuff
Agents
Build structures
Such as communication towers to commicate over long distances
Or roads?
Not restricted in movement
Structures
Summary
Structures have modular sections
Factories
Perform operations
Customize
Factory Plans
Growth
use dummy data to find good positioning?
Lead to growth of factories that produce desired input
Farms
Generate data
Uses
RNG
Energy source that can be used to limit growth
Ports
Imports/Exports data
Customize 
Port List
Factory Plans
Growth
Grow without prompt on I/O
Lead to growth of factories that produce desired input
Roads/Towers
Move data
Important structures need attachment point
Growth
Built to meet each other
Unused paths die off
Expect primary path with connected ancillary paths



\chapter{Miscellaneous}

\section{Saving Space}


Flicker Element

Flicker between elements conditionally if there is a code size limit




\end{document}